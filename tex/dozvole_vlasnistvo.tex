Važna odlike Linux i njemu sličnim sistemima je što svaki objekat(fajlovi, folderi, itd.) ima svoje dozvole za pristup. Ovaj sistem je veoma važan za održavanje visokog nivoa bezbednosti i stabilnosti Linux sistema.\\
Svaki objekat ima tri tipa dozvola: za \texttt{čitanje}(read), \texttt{pisanje}(write) i \textit{pokretanje}(execute). Ova tri tipa se odnose na dozvole: \textit{vlasnika objekta}(korsnik koji ga je napravio), \textit{grupe}(skup korsnika sa istim pravima na pristup) i na sve ostale korisnike.\\
Na svakom Linux sistemu postoji automatski napravljen administratorski korisnik ili \textit{root}, čija se prava razlikuju od ostalih korisnika, u tome što on ima puna prava na svaki objekat u sistemu. Samim tim, ima sposobnost da menja dozvole drugih kornika nad tim objektom.\\
Dozvole nekog objekta mogu da se predstave na nekoliko načina, jedan od kojih je tekstualno, preko stringa od 10 karaktera. Prvi karakter predstavlja tip objekta(\texttt{-} za običan fajl, \texttt{d} za folder). Ostalih 9 predstavlja kakve dozvole imaju vlasnik, grupa i ostali korisnici nad tim objektom(3 tipa dozvole za 3 tipa korisnika).\\
Na primer, komandom \texttt{ls -l} vidimo sve fajlove i foldere u trenutnom folderu, ali sa \texttt{-l} opcijom vidimo i karakteristike tog objekta, kao i ko ima kakve dozvole nad njim.\\
Jedan tekstualni prikaz dozvola izgleda ovako: \textit{-rwxrw-r--}. Po prvom karakteru možemo da zaključimo da je objekat fajl. Od sledeća 3 karaktera vidimo dozvole vlasnika fajla tj. da ima čitanje(\textit{r}), pisanje(\textit{w}) i pokretanje(\textit{x}) prava. Od druga 3 vidimo prava grupe i od poslednja 3 prava ostalih korisnika.\\
Vlasništvo takodje može da se prikaže brojčano tj. preko tri broja, od 0 do 8, gde svaki broj označava tip vlasništva svakog tipa korsnika. Brojevi imaju sledeća značenja:
\begin{itemize}
\item \textbf{0} - nikakva vrsta dozvola
\item \textbf{1} - samo pravo na pokretanje
\item \textbf{2} - samo pravo na pisanje
\item \textbf{3} - pravo na pokretanje i pisanje
\item \textbf{4} - samo pravo na čitanje
\item \textbf{5} - pravo na čitanje i pokretanje
\item \textbf{6} - pravo na čitanje i pisanje
\item \textbf{7} - sva prava
\end{itemize}
Prava nekog objekta mogu da se menjaju sa \texttt{chmod} komandom koristeći brojčanu metodu(prava nekog objekta jedino mogu da menjaju vlasnik fajla i \textit{root} korisnik).