Gore je već spomenuta sposobnost Linux shell-a da menja tipičan tok i destinacija output-a komandi.\\
Redirekcija je uzimanje tipičnog output-a neke komande i slati ga negde drugačije od njegove uobičajene destinacije. Na Linux sistemima, svaka komanda i samim tim, svaki proces se sadrži od tri \textit{toka podataka}: jednog ulaznog toka, standard input (ono što korinik daje) i dva izlazna toka - standard output (predvidjeni rezultat komande) i standard error (moguće greške tokom izvršavanja komande).\\
Redirekcija se sastoji od dva elementa: argument za komandu i redirekcionih operatora. Argument je fajl ili neki drugi izvor podataka koji se daje komandi da bude korišćena kao input. Na primer, u komandi \texttt{head}, koja ispisuje prvih deset linije datog teksta, argument je ili fajl sa tekstom ili neki drugi izvor teksta. \texttt{head} takodje može da se koristi bez argumenta, gde će čitati prvih deset linija koje je korisnik uneo, ali važno je napomenuti da ne može svaka komanda da se koristi bez argumenata.
\\
Redirekcioni operatori se koriste za menjanje destinacija toka podataka. Čine ih većinom tri glavna operatora: \texttt{<, >} i \texttt{|}.\\
"\texttt{>}" uzima standard output i standard error i preusmerava njihovo ispisivanje u neki fajl, za razliku od uobičajenog ispisivanja na ekranu. Na primer, komanda \texttt{cat file1 > file2}, uzima output komande \texttt{cat file1}, što bi bilo samo ispisivanje njenog sadržaja i ispisuje ga u \texttt{file2}. Ako se već neki tekst nalazio u \texttt{file2}, on će biti potpuno zamenjen sadržajem \texttt{file1}. Varijacija na \texttt{>} operator je \texttt{>>}, koji umesto da potpuno zameni sadržaj jednog fajla drugim, doda sadržaj jednog fajla na kraj drugog. Na primer, komanda \texttt{cat file1 >> file2} dodaje sadržaj od \texttt{file1} na kraj \texttt{file2}.\\
"\texttt{<}" uzima kao input komande neki fajl, umesto da čita standard input. U suštini je isto kao korišćenje fajla kao argument komande, ali je ipak vredno znati.\\
"\texttt{|}" se koristi u "piping"-u i  drugačiji je od ostalih redirekcionih operatora. Razlikuju se po tome što koriste 	output neke komande kao input druge. Na primer, output \texttt{ls} komande je lista svih fajlova i foldera u trenutnom folderu, postaje input \texttt{wc} komande koja, ako je pokrenuta bez argumenata, broji količinu redova, linija i karaktera onoga što joj je dato: 
\texttt{ls | wc}. Mogućnost spajanja komandi u jednu koristeći "piping", može znatno da skrati neki shell skript sa dosta komandi.
