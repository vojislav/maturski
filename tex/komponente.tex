\subsection{Bootloader}
Boot loader je program koji se pokreće pre bilo kog operativnog sistema. Njegov posao je da nadje operativni sistem (ili više njih) i da ga pokrene. Na Linuxu postoji nekoliko bootloader-a:\begin{itemize}
\item GRUB (\textbf{GR}and \textbf{U}nified \textbf{B}ootloader) - najpopularniji, deo GNU programa napravljenih za GNU Hurd kernel.
\item LILO (\textbf{Li}nux \textbf{Lo}ader) - razvoj je prekinut jer nije podržavao sisteme sa više od jednog operativnog sistema.
\item SYSLINUX - skup neintenzivnih bootloader-a, najčešće se koriste za podizanje sistema sa drajvova malih kapaciteta, kao fleš drajv, DVD...
\end{itemize}


\subsection{Kernel}
Kernel je najvažniji i najosnovniji deo svakog operativnog sistema. Kernel je zadužen da pokrene svaku komponentu potrebu za korišćenje sistema, da služi kao posrednik u komunikaciji izmedju softvera i hardvera, i delova softvera medjusobno. Kernel je potreban tokom celog korišćenja računara, pa je stoga neophodno da on bude što manji i što efikasniji.\\
Osnovi delove kernela su obično:
\begin{itemize}
\item rasporednik - odredjuje kako će razni procesi koristiti snagu procesora
\item supervizor - odobrava kontrolu kompjutera procesu koji je na redu
\item rukovodilac zahteva - rukuje svim zahtevima upućenim kernelu
\item menadžer memorije - dodeljuje lokacije na memoriji procesima kernela
\end{itemize}
Postji 4 glavne kategorije kernel-a:
\begin{itemize}
\item monolitski - obično se nadju kod Unix-sličnih operativnih sistema, kao kod Linux-a i FreeBSD-a. Oni sadrže sve osnove funkcije OS-a i drajvere potrebne za korišćenje hardvera kao hard diskova, grafičkih kartica, printera. Moderni monolitski kernel-i imaju opciju da odrede koji moduli kernel-a će se koristiti, time smanjujući količinu koda kernel-a.
\item microkernel-i - imaju samo minimalan broj usluga kao menadžer memorije, sistem za komunikaciju izmedju procesa i menadžer procesa. Sve ostale funkcije su implementirane nezavisno od kernel-a. Primeri mikrokernel-a su GNU Herd, MINIX i Mac OS X.
\item hibridni - kompromis izmedju monolitskih i mikrokernel-a. Osimišljeni se pre nego što je otkriveno da su mikrokernel-i daleko efikasniji od hibridnih.
Eksperimentiše se sa exokernel-ima. Glavna razlika izmedju njih i ostalih vrsta kernel-a je što se jedino bave zaštitom harvera umesto menadžmentom hardvera. Ovim pristupom exokernel-i omogućuje programerima da bolje odrede kako najefikasnije da koriste raspoloživ hardver.
\end{itemize}
\cite{kernel}

\subsection{Daemoni}
\subsection{Shell}
\subsection{X window sistem}
\subsection{Window menadzer}
\subsection{Desktop okruzenje}