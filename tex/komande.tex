Prednost GNU/Linux operativnih sistema nad drugim je ekstensivnost Linux-ovog "terminal-a". Terminal se koristi za svaku radnju na sistemu, kao manevrisanje kroz sistem, manipulacija fajlovima i folderima,  pokretanje i gašenje programa, kontrola trenutnih procesa, itd. Dalje su objašnjene neke od osnovih Linux komandi.
\begin{itemize}
\item \texttt{pwd} - Svaki put kad se otvori shell, njegov rad je koncentrisan u jedan folder (ovaj folder je pri otvaranja novog shell-a "home" folder). \texttt{pwd} ispisuje folder u kome se trenutno radi. Skraćeno od "print working direcory". 

\item \texttt{ls} - Ispisuje sve fajlove i podfoldere u zadatom folderu ili u trenutnom folderu ako ništa nije zadato. Na komande u Linux-u mogu da se dodaju opcije koje mogu dodatno da definišu tačno kakav će rezultat komande biti. Na primer, na \texttt{ls} komandu može da se doda \texttt{-a} opcija da bi se prikazali i skriveni fajlovi i folderi čija imena počinju sa ".". Skraćeno od "list".

\item \texttt{cd} - Menja folder u kome se trenutno radi na zadati folder, npr. \texttt{cd Documents}. Za vraćanje u prethodni folder koristi se \texttt{cd ..} (" .. " uvek označava folder relativno iznad trenutnog, dok " . " označava trenutni folder). Skraćeno od "change directory".

\item \texttt{mkdir} i \texttt{rmdir} - Pravi novi folder odnosno briše (pod uslovom da je prazan) zadati folder. Skraćeno od "make directory" i "remove directory".

\item \texttt{rm} - Briše zadati fajl. Može da se koristi sa opcijom \texttt{-r} (rekurzivno) da bi izbrisalo sadržaj zadatog foldera. Skraćeno od "remove".

\item \texttt{cp} - Kopira zadati fajl ili folder na datu lokaciju. Ova komanda uzima dva argumenta, odvojena razmakom. Na primer, \texttt{cp subfolder folder} ("subfolder" će se kopirati na lokaciju /folder/subfolder). Skraćeno od "copy".

\item \texttt{mv} - Premešta fajl ili folder na datu lokaciju. Takodje uzima dva argumenta. Skrećeno od "move".

\item \texttt{cat} 	- Spaja i ispisuje sadržaj fajlova (ako je dat samo jedan fajl, ispisuje njegov sadržaj). Skraćeno od "concatenate".

\item \texttt{sudo} - Stavlja se kao prefiks bilo kojoj komandi da bi se pokrenula sa administratorskim ovalašćenjima. Za pokretanje je potrebna lozinka "root" korisnika (administratora). Skraćeno od "SuperUser do".

\item \texttt{man} - Ispisuje upustva za korišćenje i sve opcije za bilo koju komandu. Skraćeno od "manual".
\end{itemize}