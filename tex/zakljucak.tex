Linux operativni sistemi su najzastupljeniji u mnogim naučnim, profesionalnim i obrazovnim institucijama. Linux je većinom standard za servere, amaterske kao i korporacijske. Preko 80\% superkompjutera koristi Linux. Zbog Android operativnog sistema, koji je zasnovan na Linux kernel-u, Linux je najzastupljeniji mobilan operativni sistem na svetu. Osim telefona, Linux je najpopularniji sistem za mikrokompjutere kao što su Raspberry Pi i Arduino mikrokontroleri.\\
GNU, koji je postao nerazdvojiv deo Linux sistema, i njegova filozofija slobode i etike softvera je donela revoluciju u svet koji je pre bio pod kontrolom nekolicine ogromnih korporacija. Linux je po svojoj prirodi besplatan i slobodan, a ne okrenut prvobitno ka profitu, ali to nije sprečilo nekoliko hiljada programera da posvete svoje živote pravljenju slobodnog softvera za širu Linux zajednicu. A to da Linux nije profitabilan je takodje osporeno postojanjem kompanija kao RedHat, koje prave slobodan softver za ogromna preduzeća, poslovna strategija koja je dovela do toga da ih IBM kupi za 34 milijarde dolara.\\
Uticaj koji je Linux imao na svet tehnologije i van toga je neizmerljiv, a on svakodnevno postoje sve veći deo naših života. Zbog Linux distribucija kao što su Ubuntu, prosečnoj osobi je lakše sad nego ikad da počne da koristi Linux, bez da zna kako ceo sistem funkcioniše, dok novi i zanimljivi projekti izlaze za one koji se više interesuju. Linux je ušao u svetu tehnologije očekivan da izumre, da bude prevazidjen i ostavljen u roku od nekoliko godina, ali sa današnjom rasprostranjenošću Linux-a i brzinom kojom se razvija, nije naivno predpostaviti da Linux ima svetlu budućnost.